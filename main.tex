\documentclass{article}

\usepackage[spanish]{babel}
\usepackage{bibentry}
\usepackage[numbers]{natbib}
\usepackage{amsthm}
\usepackage{mathtools}
\usepackage{amsmath}
\usepackage[utf8]{inputenc}

\allowdisplaybreaks

\title{Ontología} 

\begin{document}

\maketitle


\begin{abstract}

\end{abstract}

\section{Ontología}
\begin{align*}
\Sigma = \{&Spermatozoon(x), spermatozoonRoi(x,y), Midpiece(x), \\
&midpieceLenght(x,y), midpieceResidualCyto(x,y), \\
&spermMidpiece(x,y), PrincipalPiece(x), \\
&principalUniform(x,y), principalWidth(x,y), principalSharp(x,y), \\
&principalLoop(x,y), spermPrincipal(x,y), Head(x), headLeght(x,y), \\
&headAligment(x,y), headSize(x,y), headVacuoles(x,y), \\
&headContour(x), contourSmooth(x,y), contourOvalShape(x,y), \\		
&Acrosome(x),acrosomeCoverage(x,y), vacuolesQuantity(x,y), \\
&vacuolesSize(x,y), Image(x), imageDate(x,y), spermImage(x,y) \\
&Sample(x), stainingSample(x,y), Extraction(x), \\
&sampleExtraction(x,y), Patient(x), extractionPatient(x,y),\\
&Float(x),Integer(x),Boolean(x), String(x)\}
\end{align*}
Forma de la base de datos: clases y multiplicidad
\begin{align*}
Spermatozoon &\sqsubseteq \exists spermPrincipal \sqcap (\geq 1\, spermPrincipal)\\
\exists spermPrincipal &\sqsubseteq Spermatozoon \\
PrincipalPiece &\sqsubseteq \exists spermPrincipal^- \sqcap (\geq 1\, spermPrincipal^-) \\
\exists spermPrincipal^- &\sqsubseteq PrincipalPiece \\
Spermatozoon &\sqsubseteq \exists spermMidpiece \sqcap (\geq 1\, spermPrincipal)\\
\exists spermMidpiece &\sqsubseteq Spermatozoon \\
Midpiece &\sqsubseteq \exists spermMidpiece^- \sqcap(\geq 1\, spermPrincipal^-)\\
\exists spermMidpiece^- &\sqsubseteq Midpiece \\
Spermatozoon &\sqsubseteq \exists spermHead \sqcap(\geq 1\, spermHead)\\
\exists spermHead &\sqsubseteq Spermatozoon \\
Head &\sqsubseteq  \exists spermHead^- \sqcap(\geq 1\, spermHead^-)\\
\exists spermHead^- &\sqsubseteq Head \\
Head &\sqsubseteq \exists headHContour \sqcap(\geq 1\, headHContour)\\
\exists headHContour &\sqsubseteq Head\\
HContour &\sqsubseteq \exists headHContour^- \sqcap(\geq 1\, headHContour^-)\\
\exists headHcontour^- &\sqsubseteq HContour \\
Head &\sqsubseteq \exists headAcrosome \sqcap(\geq 1\, headAcrosome)\\
\exists headAcrosome &\sqsubseteq Head \\
Acrosome &\sqsubseteq \exists headAcrosome^- \sqcap(\geq 1\, headAcrosome^-)\\
\exists headAcrosome^- &\sqsubseteq Acrosome \\
\exists spermImage &\sqsubseteq Spermatozoon \\
\exists spermImage^- &\sqsubseteq Image \\
Image &\sqsubseteq \exists spermImage^- \sqcap (\geq 1\, spermImage^-)\\
\exists imageSample &\sqsubseteq Image \\
\exists imageSample^- &\sqsubseteq Sample \\
Sample &\sqsubseteq \exists imageSample^- \sqcap(\geq 1\, imageSample^-)\\
\exists sampleExtraction &\sqsubseteq Sample \\
\exists sampleExtraction ^- &\sqsubseteq Extraction \\
Extraction &\sqsubseteq \exists sampleExtraction \sqcap(\geq 1\, sampleExtraction^-)\\
\exists extractionPatient &\sqsubseteq Extraction \\
\exists extractionPatient^- &\sqsubseteq Patient \\
Patient &\sqsubseteq \exists extractionPatient ^- \sqcap(\geq 1\, extractionPatient^-)\\
NormalSpermatozoon &\sqsubseteq Spermatozoon\\
NormalPrincipalPiec &\sqsubseteq PrincipalPiece\\
nSpermNPrincipal &\sqsubseteq spermPrincipal\\
\exists nSpermNPrincipal &\sqsubseteq NormalSpermatozoon\\
\exists nSpermNPRincipal &\sqsubseteq NormalPrincipal\\
NormalMidpiece &\sqsubseteq Midpiece\\
nMidpieceNSperm &\sqsubseteq midpieceSperm\\
\exists nMidpieceNSperm &\sqsubseteq NormalMidpiece\\
\exists nMidpieceNSperm^- &\sqsubseteq NormalSpematozoon\\
NormalHead &\sqsubseteq Head\\
nHeadNSperm &\sqsubseteq headSperm\\
\exists nHeadNSperm &\sqsubseteq NormalHead\\
\exists nHeadNSperm^-& \sqsubseteq NormalSpermatozoon\\
NormalHContour &\sqsubseteq HContour\\
nHContourNHead &\sqsubseteq hContourHead\\
\exists nHContourNHead &\sqsubseteq NormalHContour\\
\exists nHContourNHead^-&\sqsubseteq NormalHead\\
NormalAcrosome &\sqsubseteq Acrosome\\
nAcrosomeNHead &\sqsubseteq acrosomeHead\\
\exists nAcrosomeNHead &\sqsubseteq NormalAcrosome\\
\exists nAcrosomeNHead^-&\sqsubseteq NormalHead
\end{align*}
Atributos de las clases (sin los tipos, por ahora)
\begin{align*}
\exists spermId &\sqsubseteq Spermatozoon\\
\exists spermRoi &\sqsubseteq Spermatozoon\\
\exists principalUniform &\sqsubseteq PrincipalPiece\\
\exists principalWidth &\sqsubseteq PrincipalPiece\\
\exists principalSharpAngle &\sqsubseteq PrincipalPiece\\
\exists principalLoop &\sqsubseteq PrincipalPiece\\
\exists principalRoi &\sqsubseteq PrincipalPiece\\
\exists midpieceLength &\sqsubseteq Midpiece\\
\exists midpieceRegular &\sqsubseteq Midpiece\\
\exists midpieceAligment &\sqsubseteq Midpiece\\
\exists midpieceResidualCytoplasm &\sqsubseteq Midpiece\\
\exists midpieceRoi &\sqsubseteq Midpiece\\
\exists headLenght &\sqsubseteq Head\\
\exists headAligment &\sqsubseteq Head\\
\exists headSize &\sqsubseteq Head\\
\exists headAspectRatio &\sqsubseteq Head\\
\exists headVacuolesQuantity &\sqsubseteq Head\\
\exists headRoi &\sqsubseteq Head\\
\exists headWidth &\sqsubseteq Head\\
\exists hcontourSmooth &\sqsubseteq HContour\\
\exists hcontourRegular &\sqsubseteq HContour\\
\exists hcontourOvalShape &\sqsubseteq HContour\\
\exists acrosomeCoverage &\sqsubseteq Acrosome\\
\exists acrosomeVacuolesQuantity &\sqsubseteq Acrosome\\
\exists acrosomeLargeVacuoles &\sqsubseteq Acrosome
\end{align*}

Página 67

The head should be smooth, regularly contoured and generally oval in shape. 
\begin{align*}
\exists hcontourSmooth.\{true\} \sqcap &\\
\exists hcontourOvalShape.\{true\} \sqcap &\\
\exists hcontourRegular.\{true\}& \equiv NormalHContour
\end{align*}

There should be a well-defined acrosomal region comprising 40–70\% of the 
head area (Menkveld et al., 2001). The acrosomal region should contain no
large vacuoles, and not more than two small vacuoles, which should not occupy
more than 20\% of the sperm head.

\begin{align*}
\exists acrosomeCoverage.< 40 \sqcap &\\
\exists acrosomeCoverage.> 70 \sqcap &\\
\exists acrosomeLargeVacuoles.\{false\} \sqcap &\\
\exists acrosomeVacuolesQuantity.< 2 \sqcap &\\
\exists acrosmeVacuoleCoverage.<20 \equiv & NormalAcrosome\\
\end{align*}
The post-acrosomal region should not
contain any vacuoles. 

The head dimensions of 77 Papanicolaou-stained spermatozoa
(stained by the procedure given in Section 2.14.2 and classified as normal by the
criteria given here), measured by a computerized system (coefficient of variation
for repeated measurements 2–7\%) had the following dimensions: median length
4.1m, 95\% CI 3.7–4.7; median width 2.8 m, 95\% CI 2.5–3.2; median length-to-width
ratio 1.5, 95\% CI 1.3–1.8.

\begin{align*}
\exists headVacuolesQuantity.0 \sqcap &\\
\exists headLength.>3.7 \sqcap &\\
\exists headLength.<4.7 \sqcap &\\
\exists headWidth.>2.5 \sqcap &\\
\exists headWidth.<3.2 \sqcap &\\
\exists headAspectRatio.>1.3 \sqcap &\\
\exists headAspectRatio.<1.8 \equiv & NormalHead
\end{align*}

(el approx es poco serio, tendría que cambiarlo o definirlo)

The midpiece should be slender, regular and about the same length as the
sperm head.

The midpieces of 74 Papanicolaou-stained spermatozoa (stained
by the procedure given in Section 2.14.2 and classified as normal by the criteria
given here) and measured by the same computerized system had the following
dimensions: median length 4.0 m, 95\% CI 3.3–5.2; median width 0.6 m, 95\% CI
0.5–0.7.

\begin{align*}
\exists midpieceRegular.\{true\} \sqcap &\\
\exists midpieceSlender.\{true\} \sqcap &\\
\exists midpieceLenght.>3.3 \sqcap &\\
\exists midpieceLenght.<5.2 \sqcap &\\
\exists midpieceWidth.>0.5 \sqcap &\\
\exists midpieceWidth.<0.7 \equiv NormalMidpiece
\end{align*}

The principal piece should have a uniform calibre along its length, be thinner
than the midpiece and be approximately 45 m long (about 10 times the head
length)\footnote{no tengo intervalos de confianza}. 
It may be looped back on itself (see Fig. 2.10c), provided there is no
sharp angle indicative of a flagellar break.

\begin{align*}
\exists principalUniform.\{true\} \sqcap &\\
\exists principalSharpAngle.\{false\} \equiv NormalPrincipalPiece
\end{align*}

Idea: formar ontologías para las características mostradas en la tabla 2.6, pág 71.

\end{document}